\documentclass[11pt]{article} % use larger type; default would be 10pt

\usepackage[utf8]{inputenc} % set input encoding (not needed with XeLaTeX)

%%% PAGE DIMENSIONS
\usepackage{geometry} % to change the page dimensions
\geometry{a4paper} % or letterpaper (US) or a5paper or....
% \geometry{margins=2in} % for example, change the margins to 2 inches all round
% \geometry{landscape} % set up the page for landscape

\usepackage{graphicx} % support the \includegraphics command and options

\usepackage[parfill]{parskip} % Activate to begin paragraphs with an empty line rather than an indent

%%% PACKAGES
\usepackage{booktabs} % for much better looking tables
\usepackage{array} % for better arrays (eg matrices) in maths
\usepackage{paralist} % very flexible & customisable lists (eg. enumerate/itemize, etc.)
\usepackage{verbatim} % adds environment for commenting out blocks of text & for better verbatim
\usepackage{subfig} % make it possible to include more than one captioned figure/table in a single float
% These packages are all incorporated in the memoir class to one degree or another...

%%% HEADERS & FOOTERS
\usepackage{fancyhdr} % This should be set AFTER setting up the page geometry
\pagestyle{fancy} % options: empty , plain , fancy
\renewcommand{\headrulewidth}{0pt} % customise the layout...
\lhead{}\chead{}\rhead{}
\lfoot{}\cfoot{\thepage}\rfoot{}

%%% SECTION TITLE APPEARANCE
\usepackage{sectsty}
\allsectionsfont{\sffamily\mdseries\upshape} % (See the fntguide.pdf for font help)
% (This matches ConTeXt defaults)

%%% ToC (table of contents) APPEARANCE
\usepackage[nottoc,notlof,notlot]{tocbibind} % Put the bibliography in the ToC
\usepackage[titles,subfigure]{tocloft} % Alter the style of the Table of Contents
\renewcommand{\cftsecfont}{\rmfamily\mdseries\upshape}
\renewcommand{\cftsecpagefont}{\rmfamily\mdseries\upshape} % No bold!

%%% END Article customizations

%%% The "real" document content comes below...

\title{Sources of Non-Determinism in Program Execution}
\author{Syed Raza}
%\date{} % Activate to display a given date or no date (if empty),
         % otherwise the current date is printed 

\begin{document}

\maketitle

\section{Introduction}
Take a single program, and execute it several times. Does the program always execute
the same sequence of instructions? If not, what can cause multiple executions of the same program 
to diverge in behavior? 

One can immediately find obvious reasons for multiple executions of the same program to be different: instructions in multi-threaded 
applications can be interleaved in several ways; inputs or data can differ; random number
generators are frequently used, and so on.
Surprisingly, however, even a simple ``Hello, world!" application in C does not
always produce the same execution trace.

This paper describes sources of non-determinism  -- obvious and obscure --
for certain classes of programs, along with strategies to make program execution 
deterministic. Such a study of application behavior is novel and
interesting in its own right; it reveals important interactions between
applications, underlying operating systems and intermediate software layers,
and provides insight into common application behavior.

\section{Terminology}
In this paper, an application's \emph{execution trace} refers to instructions executed, in user mode,
by the process for the application, along with the side-effects of those instructions (e.g. values read/written from registers or memory).

Two execution traces can thus be different if they contain different instructions (e.g. control flow is not identical),
or if the same instructions have different side effects (e.g. application memory is not identical).

We use dynamic binary instrumentation tools like DynamoRio and Pin to collect execution traces from 
applications, and to modify application behavior to overcome any sources of non-determinsm.

\section{Before entering \texttt{main()}}
This section explains why execution traces can diverge even before programs
have started executing. All these factors combine to explain why 
even a simple "Hello, World!" program in C can generate many different
traces.

\subsection{Address Space Layout Randomization (ASLR)}
Implemented by several mainstream operating systems, Address Space Layout Randomization (ASLR) is a security feature
which involves random arrangement of the positions of key data areas for an executing program. For instance, the 
base of the executable, the position of libaries, heap and the stack in a process's address space can be 
different every time the program is run.

ASLR hinders several kinds of security attacks because it makes it difficult for attackers to predict
addresses for redirecting execution. For instance, \emph{return-to-libc} attacks usually start with a buffer overflow attack
which replaces the return address on the stack with the address of another existing instruction and overwrites an additional
portion of the stack to provide arguments to this function. Because the common
C runtime on UNIX systems (``libc'') is always linked with programs, and contains useful calls (e.g. \texttt{system()})
to run arbirtrary programs, most attacks simply provide one argument on the stack and execute a libc function like \texttt{system()}.
If libc functions are always placed at the same addresses, attackers can avoid injecting malicious code,
and just redirect execution to them.

Non executable stacks cannot prevent return-to-libc attacks. While stack-smashing protection can help prevent
such attacks, ASLR is the most effective in making return-to-libc attacks very difficult on 64-bit machines. On 32-bit
machines, new attacks reduce the effectiveness of ASLR to searching from a 16-bit space, which can be
done in a matter of minutes.

Because of ASLR, two execution traces of a simple "Hello, World!" C program are entirely different (i.e. they
have no overlap). Almost every instruction has a different address, and stack addresses are clearly different too.

Because ASLR is not effective in preventing return-to-libc attacks on 32-bit machines, ASLR can be disabled
to make program execution more deterministic. Most distributions of Linux allow ASLR to be turned off via
the following command:

\texttt {sudo kernel.randomize\_va\_space=0}

Disabling ASLR greatly increases the overlap between different traces of the "Hello,World!" program,
but the traces are not identical yet.

\subsection{Stack Protection: \texttt{libc} Canary} 

\subsection{Pointer Encryption} 

\subsection{Environment Variables} 

\subsection{Elf Auxiliary Vectors} 

\subsection{Command Line Arguments} 

\subsection{Thread Local Storage} 

\section{Inside \texttt{main()}}

\subsection{\texttt{rdtsc} instruction} 

\subsection{\texttt{cpuid} instruction} 

\subsection{\texttt{getpid()} system call} 

\subsection{reads from \texttt{/dev/urandom}}
 
\subsection{reading the current time}
 
\end{document}
