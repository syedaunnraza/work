% $Log: abstract.tex,v $
% Revision 1.1  93/05/14  14:56:25  starflt
% Initial revision
% 
% Revision 1.1  90/05/04  10:41:01  lwvanels
% Initial revision
% 
%
%% The text of your abstract and nothing else (other than comments) goes here.
%% It will be single-spaced and the rest of the text that is supposed to go on
%% the abstract page will be generated by the abstractpage environment.  This
%% file should be \input (not \include 'd) from cover.tex.
Server virtualization enables data centers to run many VMs on individual hosts -- 
this reduces costs, simplifies administration and facilitates management.
Improvements in hardware and virtualization technology,
coupled with the use of virtualization for desktop machines with modest 
steady-state resource utilization, are expected to allow
individual hosts to run thousands of VMs at the same time.
Such high {\em VM densities per host} would allow data centers
to reap unprecedented cost-savings in the future.

Unfortunately, hosts running many VMs 
can be throttled due to unusually high CPU and
memory pressure generated when such VMs 
boot up concurrently. Over provisioning hardware
to avoid prohibitively high boot latencies that result from
these -- often daily -- {\em boot storms} 
is clearly expensive.

The aim of this thesis is to investigate whether a hypervisor could theoretically exploit the overlap in 
the instruction streams of concurrently booting VMs to reduce CPU pressure in boot storms. 
This idea, which we name {\em silhouette execution}, would allow hypervisors
to use the CPU in a scalable way, much like {\em transparent page sharing}
allows a hypervisor to use its limited memory in a scalable fashion.

To evaluate silhouette execution, we studied user-space
instruction streams from a few Linux services using dynamic instrumentation.
We statistically profiled the extent of nondeterminism in program execution, and   
compiled the reasons behind any execution differences.
Though there is significant overlap in the user-mode instruction streams
of Linux services, our simple simulations show that silhouette
execution would {\em increase} CPU pressure by 13\% for 100 VMs and 6\% for 1000 VMs.
We present a few strategies for reducing synthetic differences in execution in user-space
programs; our simulations show that silhouette execution can reduce CPU pressure on a host by a factor
of $8\times$ for 100 VMs and a factor of $19\times$ for 1000
VMs once these strategies are used. We believe that the insights provided in this thesis on controlling
execution differences in concurrently booting VMs via
dynamic instrumentation are a prelude to a successful future implementation of {\em silhouette execution}.




