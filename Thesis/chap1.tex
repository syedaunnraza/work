%% This is an example first chapter.  You should put chapter/appendix that you
%% write into a separate file, and add a line \include{yourfilename} to
%% main.tex, where `yourfilename.tex' is the name of the chapter/appendix file.
%% You can process specific files by typing their names in at the 
%% \files=
%% prompt when you run the file main.tex through LaTeX.
\chapter{Introduction}
\section{Overview}
Take a single program, and execute it several times. Does the 
program always execute the same sequence of instructions?
If not, what can cause multiple executions of the same program
be different? Do the results change if we run
the same program in separate 
{\em e.g.} two identical but distinct virtual machines,
or two slightly different virtual machines?

One can immediately find a few obvious reasons
for multiple executions of the same program
to behave differently: input or data can vary;
many programs deliberately use randomization schemes;
instructions in multithreaded programs
can interleave in myriad ways, and so on.
Surprisingly, however, even a simple ``Hello, world!'' 
application written in C does not execute the 
same sequence of instructions.

To the best of our knowledge, the sources of nondeterminism 
in modern applications have not been studied in great detail or documented
in a single place. This thesis synthesizes sources of nondeterminism  -- both
obvious and obscure -- for a select group of programs, and describes the design
and implementation of a tool that makes program execution as deterministic as possible.
Despite its applications, such a study is novel and interesting in its own right,
for it reveals subtle interactions 
between applications, underlying operating systems 
and intermediate software layers.

\section{Importance of Deterministic Execution}\label{ch1:whydeterminism}
Deterministic execution of programs can be beneficial in many
different scenarios, and has motivated the design of Kendo,
a system which enables deterministic multithreading
in applications\cite{}. Our tool complements Kendo because 
it focuses on deterministic execution of single-threaded services in Linux,
at the granularity of individual instructions and their side-effects in
memory.

The motivations for deterministic multhreading listed in
\cite{} apply to this thesis as well.

\subsection{Mainstream Computing and Security}


\subsection{Repeatability}
Users expect programs to produce the same outputs, given the same
inputs. For safety-critical systems, it may
be even better if we can guarantee that given the same inputs,
a program would always execute some {\em precise} 
set of instructions, in the same order, with
the same side-effects. Record/replay systems are not suitable for 
achieving such strong guarantees of repeatability, given the storage
overhead for logs and the typically enormous input space. 
Our system, like Kendo, requires minimal storage to achieve
single-threaded determinism. 

\subsection{Debugging}
Determinism is important for debugging, because developers
often need to reproduce erroneous behavior in
order to diagnose and fix it. Nondeterminism is typically
a major issue for debugging multithreaded applications,
and is a lesser issue for single-threaded
applications. Our work will help developers
easily reproduce erroneous behavior given
the same input for single-threaded applications. 
Record/replay systems have high overhead, so it is unlikely
that the initial buggy execution of a program was recorded.
Deterministic execution also precludes the need for storing
many gigabytes of logs needed for record/replay.

\subsection{Testing}
Deterministic execution in general facilitates testing,
because outputs and internal state can be checked at 
certain points with respect to expected values. Our version
of determinism allows for a particularly strong kind
of test case that may be necessary for safety-critical 
systems: with deterministic execution, a program 
must execute the exact same instructions 
across different executions, for the same inputs.
Test cases can check for deviations from 
the expected instruction sequences.

\section{The VM {\em Bootstorm} Problem}\label{ch1:bootstorm}


% This is an example of how you would use tgrind to include an example
% of source code; it is commented out in this template since the code
% example file does not exist.  To use it, you need to remove the '%' on the
% beginning of the line, and insert your own information in the call.
%
%\tagrind[htbp]{code/pmn.s.tex}{Post Multiply Normalization}{opt:pmn}

% This is an example of how you would use tgrind to include an example
% of source code; it is commented out in this template since the code
% example file does not exist.  To use it, you need to remove the '%' on the
% beginning of the line, and insert your own information in the call.
%
%\tgrind[htbp]{code/be.s.tex}{Block Exponent}{opt:be}

\section{Contributions}

As well as the floating point optimizations described above, there are
also integer optimizations that can be used in the $\mu$FPU.  In concert
with the floating point optimizations, these can provide a significant
speedup.  

\subsection{Organization}

Integer operations are much faster than floating point operations; if it is
possible to replace floating point operations with fixed point operations,
this would provide a significant increase in speed.

This conversion can either take place automatically or or based on a
specific request from the programmer.  To do this automatically, the
compiler must either be very smart, or play fast and loose with the accuracy
and precision of the programmer's variables.  To be ``smart'', the computer
must track the ranges of all the floating point variables through the
program, and then see if there are any potential candidates for conversion
to floating point.  This technique is discussed further in
section~\ref{range-tracking}, where it was implemented.

The other way to do this is to rely on specific hints from the programmer
that a certain value will only assume a specific range, and that only a
specific precision is desired.  This is somewhat more taxing on the
programmer, in that he has to know the ranges that his values will take at
declaration time (something normally abstracted away), but it does provide
the opportunity for fine-tuning already working code.

Potential applications of this would be simulation programs, where the
variable represents some physical quantity; the constraints of the physical
system may provide bounds on the range the variable can take.
\subsection{Small Constant Multiplications}

One other class of optimizations that can be done is to replace
multiplications by small integer constants into some combination of
additions and shifts.  Addition and shifting can be significantly faster
than multiplication.  This is done by using some combination of
\begin{eqnarray*}
a_i & = & a_j + a_k \\
a_i & = & 2a_j + a_k \\
a_i & = & 4a_j + a_k \\
a_i & = & 8a_j + a_k \\
a_i & = & a_j - a_k \\
a_i & = & a_j \ll m \mbox{shift}
\end{eqnarray*}
instead of the multiplication.  For example, to multiply $s$ by 10 and store
the result in $r$, you could use:
\begin{eqnarray*}
r & = & 4s + s\\
r & = & r + r
\end{eqnarray*}
Or by 59:
\begin{eqnarray*}
t & = & 2s + s \\
r & = & 2t + s \\
r & = & 8r + t
\end{eqnarray*}
Similar combinations can be found for almost all of the smaller
integers\footnote{This optimization is only an ``optimization'', of course,
when the amount of time spent on the shifts and adds is less than the time
that would be spent doing the multiplication.  Since the time costs of these
operations are known to the compiler in order for it to do scheduling, it is
easy for the compiler to determine when this optimization is worth using.}.
\cite{magenheimer:precision}

\section{Other optimizations}

\subsection{Low-level parallelism}

The current trend is towards duplicating hardware at the lowest level to
provide parallelism\footnote{This can been seen in the i860; floating point
additions and multiplications can proceed at the same time, and the RISC
core be moving data in and out of the floating point registers and providing
flow control at the same time the floating point units are active. \cite{byte:i860}}

Conceptually, it is easy to take advantage to low-level parallelism in the
instruction stream by simply adding more functional units to the $\mu$FPU,
widening the instruction word to control them, and then scheduling as many
operations to take place at one time as possible.

However, simply adding more functional units can only be done so many times;
there is only a limited amount of parallelism directly available in the
instruction stream, and without it, much of the extra resources will go to
waste.  One process used to make more instructions potentially schedulable
at any given time is ``trace scheduling''.  This technique originated in the
Bulldog compiler for the original VLIW machine, the ELI-512.
\cite{ellis:bulldog,colwell:vliw}  In trace scheduling, code can be
scheduled through many basic blocks at one time, following a single
potential ``trace'' of program execution.  In this way, instructions that
{\em might\/} be executed depending on a conditional branch further down in
the instruction stream are scheduled, allowing an increase in the potential
parallelism.  To account for the cases where the expected branch wasn't
taken, correction code is inserted after the branches to undo the effects of
any prematurely executed instructions.

\subsection{Pipeline optimizations}

In addition to having operations going on in parallel across functional
units, it is also typical to have several operations in various stages of
completion in each unit.  This pipelining allows the throughput of the
functional units to be increased, with no increase in latency.

There are several ways pipelined operations can be optimized.  On the
hardware side, support can be added to allow data to be recirculated back
into the beginning of the pipeline from the end, saving a trip through the
registers.  On the software side, the compiler can utilize several tricks to
try to fill up as many of the pipeline delay slots as possible, as
seendescribed by Gibbons. \cite{gib86}


