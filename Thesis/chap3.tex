%% This is an example first chapter.  You should put chapter/appendix that you
%% write into a separate file, and add a line \include{yourfilename} to
%% main.tex, where `yourfilename.tex' is the name of the chapter/appendix file.
%% You can process specific files by typing their names in at the 
%% \files=
%% prompt when you run the file main.tex through LaTeX.
\newenvironment{mylisting}
{\begin{list}{}{\setlength{\leftmargin}{1em}}\item\scriptsize\bfseries}
{\end{list}}

\chapter{Overcoming Nondeterminism in Linux services} \label{ch:src}
This chapter summarizes the sources of nondeterminism
in Linux services discovered through our experiments (Section \ref{ch3:sources}).
It also explains the techniques we used to simulate 
deterministic execution (Section \ref{ch3:simulation}). 
Finally, it briefly discusses some of the inherent
limitations of deterministic execution (Section \ref{ch3:issues}).

\section{Sources of Nondeterminism} \label{ch3:sources}
In this section, we describe the sources of nondeterminism
discovered using the data collection scheme
described in Chapter \ref{ch:boot}.
This study of nondeterminism reveals
subtle interactions between user-mode
applications, commonly used system libraries (e.g. the \texttt{libc} library),
the Linux operating system and the external world.
While our results are derived from analyzing a small
set of complex programs, they include
all sources of application-level nondeterminism that 
have been described in literature. Unlike existing work,
however, we cover the various interfaces between user-mode programs
and the Linux kernel in considerable detail.

\subsection{Linux Security Features} \label{ch3:security}
\noindent {\bf Address Space Layout Randomization (ASLR)} \newline
Address Space Layout Randomization (ASLR) involves random arrangement of
key memory segments of an executing program. When ASLR is enabled,
virtual addresses for the base executable, shared libraries, 
the heap, and the stack are different across multiple executions.
ASLR hinders several kinds of security attacks in which attackers have to predict
program addresses in order to redirect execution (e.g. \emph{return-to-libc} attacks). 
As mentioned earlier, two execution traces of even a
simple program in $C$ are almost entirely different
when ASLR is enabled because of different
instruction and memory addresses. \newline

\noindent {\bf Canary Values and Stack Protection} \newline
Copying a \emph{canary} -- a dynamically chosen global value -- onto the
stack before each function call can help detect buffer overflow attacks, because 
an attack that overwrites the return address will also overwrite
a copy of the canary. Before a \texttt{ret}, a simple comparison 
of the global (and unchanged) canary
value with the (possibly changed) stack copy can prevent a buffer overflow attack.

In 32-bit Linux distributions, the $C$ runtime library, 
\texttt{libc}, provides a canary value in \texttt{gs:0x14}.
If Stack Smashing Protection (SSP) is enabled on compilation,
\texttt{gcc} generates instructions that use the canary value
in \texttt{gs:0x14} to detect buffer overflow attacks.
Because Pin gets control of the application before \texttt{libc}
initializes \texttt{gs:0x14}, multiple execution traces of a program
will diverge when \texttt{gs:0x14} is initialized and subsequently
read.  The manner in which the canary value in \texttt{gs:0x14} is initialized
depends on the \texttt{libc} version.
If randomization is disabled, \texttt{libc} will store a fixed
terminator canary value in \texttt{gs:0x14}; this does not lead to any nondeterminism.
When randomization is enabled, however,  
some versions of \texttt{libc} store an unpredictable value in \texttt{gs:0x14} 
by reading from \texttt{`/dev/urandom'}
or by using the \texttt{AT\_RANDOM} bytes provided by the kernel (see
Section \ref{ch3:rand}). 

\newpage
\noindent {\bf Pointer Encryption} \newline
Many stateless APIs return data pointers to clients 
that the clients are supposed to supply as arguments
to subsequent function calls. 
For instance, the \texttt{setjmp} and \texttt{longjmp} functions
can be used to implement a try-catch block in $C$: \texttt{setjmp} uses 
a caller-provided, platform-specific \texttt{jmp\_buf} structure
to store important register state that \texttt{longjmp} 
later reads to simulate a return from \texttt{setjmp}.
Since the \texttt{jmp\_buf} instance is accessible to clients of \texttt{setjmp}
and \texttt{longjmp}, it is possible that the clients may advertently or inadvertently
overwrite the return address stored in it and simulate a buffer-overflow attack
when \texttt{longjmp} is called.

Simple encryption schemes can detect mangled data structures.
For instance, in 32-bit Linux, \texttt{libc} provides
a {\em pointer guard}  in \texttt{gs:0x18}. 
The idea behind the pointer guard is the following: 
to encrypt a sensitive address $p$, a program
can compute $s = p$  $\oplus  $ \texttt{gs:0x18}, 
optionally add some bit rotations, and store it in a structure
that gets passed around. Decryption can simply invert any bit rotations, 
and then compute $p = s$ $\oplus  $ \texttt{gs:0x18} back. 
Any blunt writes to the structure from clients will be detected because
decryption will likely not produce a valid pointer. 
Pointer encryption is a useful security feature for some APIs
and is used by some versions of \texttt{libc} to protect addresses stored in \texttt{jmp\_buf}
structures.

The \texttt{libc} pointer guard has different values
across multiple runs of a program, just like the canary
value. Initialization of the \texttt{libc} pointer guard can 
therefore be a source of nondeterminism in program execution. 
In some versions of \texttt{libc}, the value of \texttt{gs:0x18} is the same
as the value of \texttt{gs:0x14} (the canary). In others,
the value of \texttt{gs:0x18} is computed by \texttt{XOR}ing \texttt{gs:0x14} with 
a random word (e.g. the return value of the \texttt{rdtsc} x86 instruction),
or reading other \texttt{AT\_RANDOM} bytes provided by the kernel
 (Section \ref{ch3:rand}).

\subsection{Randomization Schemes} \label{ch3:rand}
As already clear from Section \ref{ch3:security}, 
randomization schemes can lead to significant nondeterminism 
in programs. Applications generally employ pseudorandom number
generators (PRNGs), so they need only a few random
bytes to {\em seed} PRNGs. 

These few random bytes are
usually read from one of few popular sources:
\begin{itemize}
\item {\em The} \texttt{`/dev/urandom'}{\em special file}. Linux allows
running processes to access a random number generator through this
special file. The entropy generated from environmental noise (including
device drivers) is used in some implementations of the kernel random number generator.

\item \texttt{AT\_RANDOM} {\em bytes}.
Using \texttt{open}, \texttt{read} and \texttt{close} system-calls 
to read only a few random bytes from \texttt{`/dev/urandom'} 
can be computationally expensive. 
To remedy this, some 
recent versions of the Linux kernel supply
a few random bytes to all executing programs
through the \texttt{AT\_RANDOM} auxiliary vector.
ELF auxiliary vectors are pushed on the program
stack before a program starts executing below command-line arguments and environmental
variables.

\item {\em The} \texttt{rdtsc} {\em instruction}.
The \texttt{rdtsc} instruction provides an approximate number of ticks since
the computer was last reset, which is stored in a 64-bit register present
on x86 processors. Computing the difference between two successive
calls to \texttt{rdtsc} can be used for timing whereas a single
value returned from \texttt{rdtsc} lacks any useful context.  
The instruction has low-overhead, which makes it suitable for generating a random value
instead of reading from \texttt{`/dev/urandom'}. 

\item {\em The current time or process ID}. 
System calls that return the current
process ID (Section \ref{ch3:pid}) or time (Section \ref{ch3:time})
generate unpredictable values across
executions, and are commonly used to seed PRNGs.

\item {\em Miscellaneous}: There
are several creative ways to seed PRNGs, including 
using {\em www.random.org}
or system-wide performance statistics.
Thankfully, we have not observed them 
in our analysis of Linux services.
\end{itemize}

Randomization-related nondeterminism thus usually originates from
any external sources used to seed PRNGs;
if the seeds are different across multiple
executions, PRNGs further propagate this
nondeterminism.

\subsection{Process Identification Layer} \label{ch3:pid}
% pid, signal(pid), /proc/ filesystem layer

In the absence of a deterministic operating system layer, process IDs for
programs are generally not predictable.
For instance, a nondeterministic scheduler (Section \ref{ch3:concurrency}) 
could lead to several possible process creation sequences
and process ID assignments when a VM boots up.
 
Given the unpredictability of process IDs,
system calls that directly or indirectly
interact with the process identification layer can cause divergences
across distinct executions of the same program.
For instance, system calls that return a process ID e.g.
\texttt{getpid} (get process ID), \texttt{getppid} (get
parent process ID), \texttt{fork/clone} (create a child process),
\texttt{wait} (wait for a child process to terminate)
return conflicting values across distinct executions. System calls that take process IDs 
directly as arguments such as \texttt{kill} (send a signal to a specific
process), \texttt{waitpid} (wait for a specific child process to terminate)
can similarly propagate any nondeterminism.
In fact, \texttt{libc} stores a copy of the current process ID in \texttt{gs:0x48},
so reads from this address also propagate execution differences.

Apart from system calls, there are other interfaces
between the Linux kernel and executing user-mode programs
where process IDs also show up:

\begin{itemize} 

\item {\em Signals}: If a process registers a signal handler with the \texttt{SA\_SIGINFO}
bit set, then the second argument passed
to the signal handler when a signal occurs is of type \texttt{siginfo\_t*}.
The member \texttt{siginfo\_t.si\_pid} will
be set if another process sent the signal 
to the original process (Section \ref{ch3:sig}). 

\item {\em Kernel messages}: The Linux kernel will sometimes use process IDs 
to indicate the intended recipients of its messages. 
For instance, \texttt{Netlink} is a socket-like
mechanism for inter process communications (IPC)
between the kernel and user-space processes.
\texttt{Netlink} can be used to pass
networking information between kernel
and user-space, and some of its APIs 
use process IDs to identify communication
end-points (Section \ref{ch3:netio}). \end{itemize}

\newpage 
Nondeterminism arising from the unpredictability of process IDs can be
further propagated when an application uses process IDs to seed PRNGs 
(Section \ref{ch3:rand}), access the \texttt{`/proc/[PID]'} directory
(Section \ref{ch3:procfs}), name application-specific files
(e.g. \texttt{`myapp-[pid].log'}) or log some information to files
(e.g. \emph{`process [pid] started at [04:23]'}) (Section \ref{ch3:fileio}). 

\subsection{Time} \label{ch3:time}
Concurrent runs of the same program will typically
execute instructions at (slightly) different times.
Clearly, any interactions of a program with timestamps
can cause nondeterminism. For instance:

\begin{itemize}
\item The \texttt{time}, \texttt{gettimeofday} and \texttt{clock\_gettime}
 system calls return the current time.
\item The \texttt{times} or \texttt{getrusage} system calls
return process and CPU time statistics respectively.
\item The \texttt{adjtimex} system call is used 
by clock synchronization programs (e.g. \texttt{ntp}) 
and returns a kernel timestamp indirectly via 
a \texttt{timex} structure.
\item Programs can access the hardware clock
through \texttt{`/dev/rtc'} and read the current time
through the \texttt{RTC\_RD\_TIME} \texttt{ioctl}
operation.
\item Many system calls that specify a timeout
for some action (e.g. \texttt{select}, \texttt{sleep} or \texttt{alarm})
inform the caller of any unused time from the timeout interval if they
return prematurely.
\item The \texttt{stat} family of system calls returns file
  modification timestamps; also, many application files typically contain timestamps;
  network protocols use headers with timestamps as well (Sections \ref{ch3:fileio}
  and \ref{ch3:netio}).
\end{itemize}

Apart from nondeterminism arising
from timestamps, {\em timing} differences
can arise between distinct executions 
because of variable system-call latencies 
or unpredictable timing of
external events relative
to program execution (Sections \ref{ch3:sig} and \ref{ch3:poll}).

\subsection{File I/O} \label{ch3:fileio}
\noindent {\bf File contents} \newline
If two executions of the same program read different
file contents (e.g. cache files), then
there will naturally be execution divergence.
For concurrently executing Linux services,
differences in file contents typically arise
from process IDs (Section \ref{ch3:pid}) or timestamps (Section \ref{ch3:time})
rather than semantic differences.
Once those factors are controlled, file contents rarely differ. \newline

\noindent {\bf File Modification Times} \newline
Apart from minor differences in file contents,
nondeterminism can arise from distinct file 
modification (\texttt{mtime}), access (\texttt{atime}) or status-change (\texttt{ctime})
timestamps.
The \texttt{stat} system call is usually made for almost
every file opened by a program; the time values
written by the system call invariably
conflict between any two executions. Most of the time,
these timestamps are not read by programs,
so there is little propagation. On occasion, 
however, a program will use these timestamps
to determine whether a file is more recent than another,
or whether a file has changed since
it was last read. \newline

\noindent {\bf File Size} \newline
When a program wishes to open
a file in append-mode, it uses \texttt{lseek}
with \texttt{SEEK\_END} to move
the file cursor to the end,
before any \texttt{write}s take place.
The return value of \texttt{lseek} is the
updated cursor byte-offset into the file.
Clearly, if the length of a file is different across
multiple exections of a program, then
\texttt{lseek} will return conflicting values.
Many Linux services maintain log files
which can have different lengths due
to conflicts in an earlier execution; \texttt{lseek}
further propagates them. To overcome
such nondeterminism, older log files
must be identical at the beginning 
of program execution and other
factors that cause nondeterminism
must be controlled. 

\newpage
%Ultimately, however, if two input or configuration files
%are semantically different between
%different executions of a program, then 
%execution will inevitably diverge. 

\subsection{Network I/O} \label{ch3:netio}
\noindent {\bf Network Configuration Files} \newline
The \texttt{libc} network initialization
code loads several configuration files
into memory (e.g. \texttt{`/etc/resolv.conf'}). 
Differences in the content, timestamps or lengths
of such configuration files can clearly cause nondeterminism.
Background daemons (e.g. \texttt{dhclient} for \texttt{`/etc/resolv.conf'}) 
usually update these files periodically in the background.
Calls to \texttt{libc} functions such as \texttt{getaddrinfo} use \texttt{stat} 
to determine if relevant configuration files (e.g. \texttt{`/etc/gai.conf'})
have been modified since they were last read. 
In our experiments, typically the file modification timestamps 
-- and not the actual contents -- of these configuration files 
vary between different executions. \newline

\noindent {\bf DNS Resolution} \newline
In our experiments, IP addresses are resolved identically by concurrently executing services. 
However, if DNS-based load-balancing schemes are used, the same 
server can appear to have different IP addresses. \newline

\noindent {\bf Socket \texttt{read}s} \newline
Bytes read from sockets can differ between
executions for a variety
of reasons. For instance, different timestamps in 
protocol headers, or different requests/responses 
from the external world would be reflected in
conflicting socket \texttt{read}s.
By studying application behavior, it is possible
to distinguish between 
these different scenarios and
identify the seriousness
of any differences in the bytes read.

In our experiments, we observed nondeterminism
in \texttt{read}s from \texttt{Netlink} sockets.
As mentioned in Section \ref{ch3:pid},
\texttt{Netlink} sockets provide a 
mechanism for inter-process communications (IPC)
between the kernel and user-space processes.
This mechanism can be used to pass
networking information between kernel
and user space. \texttt{Netlink} sockets
use process IDs to identify
communication endpoints, which can 
differ between executions (Section \ref{ch3:pid}).
Similarly, some implementations of \texttt{libc} use
timestamps to assign monotonically increasing sequence 
numbers to \texttt{Netlink} packets (Section \ref{ch3:time}).
Nondeterminism can also arise from sockets of the \texttt{NETLINK\_ROUTE}
family, which receive routing and link updates
from the kernel; \texttt{libc} receives \texttt{RTM\_NEWLINK}
messages when new link interfaces 
in the computer are detected. When an interface
gets discovered or reported, the kernel supplies
interface statistics to \texttt{libc} 
such as packets sent, dropped or
received. These statistics will obviously vary
across different program instances.  
\newline

\noindent {\bf Ephemeral Ports} \newline
A TCP/IPv4 connection consists of two end-points;
each end-point consists of an IP address and a port
number. An established client-server connection 
can be thought of as the
4-tuple (server\_IP, server\_port, client\_IP, client\_port).
Usually three of these four are readily known:
a client must use its own IP, and
the pair (server\_IP, server\_port) is fixed. What is not
immediately evident is that the client-side of 
the connection uses a port number.
Unless a client program explicitly
requests a specific port number,
an {\em ephemeral port} is used.
Ephemeral ports are temporary ports that are
assigned from a dedicated range by the machine IP stack.
An ephemeral port can be recycled
when a connection is terminated.
Since the underlying operating system
is not deterministic, ephemeral
port numbers used by Linux services
tend to be different across multiple 
runs. 

\subsection{Scalable I/O Schemes} \label{ch3:poll}
\noindent {\bf Polling Engines} \newline
Complex programs like Linux services have
many file descriptors open at a given time.
Apart from regular files, these special 
file descriptors could correspond to:

\begin{itemize}

\item {\em Pipes}: Pipes are used for
  one-way interprocess communication (IPC).
  Many Linux services spawn child processes;
  these child processes communicate
  with the main process (e.g. for status
  updates) through pipes.

\item A {\em listener socket}:
  If the program is a server,
  this is the socket that accepts incoming connections.

\item {\em Client-handler sockets}:
  If this program is a server, 
  new requests from already connected clients would arrive through
  these sockets.

\item {\em Outgoing sockets}:
  If the program is a client for other servers,
  it would use these sockets to send requests 
  to them.
\end{itemize}

The classic paradigm for implementing server
programs is {\em one thread or process per client}
because I/O operations are traditionally blocking
in nature. This approach scales poorly as the number
of clients -- or equivalently, the number of open special file descriptors -- increases. 
As an alternative, event-based I/O is increasingly used 
by scalable network applications.
In such designs, the main event-thread
specifies a set of file descriptors it cares about,
and then waits for ``readiness'' notifications 
from the operating system on any of
these file descriptors by using a
system call such as \texttt{epoll}, \texttt{poll},
\texttt{select} or \texttt{kqueue}. 
For instance, a client socket would be ready 
for reading if new data was received from a client,
and an outgoing socket would be ready for 
writing if an output buffer was flushed out or if the 
connection request was accepted.
The event-thread invokes an I/O
handler on each received event, 
and then repeats the loop to process the next
set of events.
This approach is often used for design simplicity because it
reduces the threads or processes needed by an application; 
recent kernel implementations (e.g. \texttt{epoll}) are also
efficient because they return the set of file descriptors that are ready for I/O,
preventing the need for the application to iterate through all its open
file descriptors. 

Event-based I/O can be a source
of nondeterminism in programs because the
timing of I/O events with respect to each other
can be different across multiple executions.
Even if I/O events are received in the same order,
the same amount of data may not be available
from ready file descriptors. Furthermore, when a timeout
interval is specified by the application for polling file descriptors,
\texttt{select} may be completed or interrupted
prematurely. In that case, \texttt{select} returns
the remaining time interval, which can     
differ between executions (Section \ref{ch3:time}). \newline

\noindent {\bf Asynchronous I/O Systems} \newline
Asynchronous I/O APIs (e.g. the Kernel Asynchronous I/O interface
in some Linux distributions) allow even a single application
thread to overlap I/O operations with other processing
tasks. A thread can request an I/O operation (e.g. \texttt{aio\_read}),
and later query the operating system for its status or ask to be notified when the I/O operation
has been completed (e.g. \texttt{aio\_return}). While such APIs are in limited
usage, they introduce nondeterminism because of the
variable latency and unpredictable relative timing of I/O events.

\subsection{Signals}\label{ch3:sig}
A signal is an event generated by Linux
in response to some condition, which may cause
a process to take an action in response.
Signals can be generated by error conditions
(e.g. memory segment violations), 
terminal interrupts (e.g. from the shell), 
inter-process communication (e.g. parent 
sends \texttt{kill} to child process),
or scheduled \texttt{alarm}s. 
Processes register handlers (or function callbacks) for specific signals
of interest in order to respond to them.

Signals are clearly external to
instructions executed by a single process,
as such, they create nondeterminism 
much the same way as asynchronous I/O:
signals can be delivered to multiple executions
of the same program in different order; 
even if signals are received in the
same order between different executions,
they can be received at different times
into the execution of a program.

\subsection{Concurrency} \label{ch3:concurrency}
Multiple possible instruction-level interleavings of 
threads within a single program, 
or of different processes within 
a single operating system are
undoubtedly significant sources
of nondeterminism in programs.
Nondeterminism due to
multi-threading has been extensively
documented and can cause
significant control flow differences
across different executions of the same
program.
\newpage
Nondeterminism in the system
scheduler is external to 
program execution, and manifests itself
in different timing or ordering of
inter-process communication e.g.
through pipes (Section \ref{ch3:poll}), signals (Section \ref{ch3:sig}), or 
values written to shared files or logs (Section \ref{ch3:fileio}).

\subsection{{\em Procfs}: The `/proc/' directory}\label{ch3:procfs}
Instead of relying on system-calls, user-space programs
can access kernel data much more easily using {\em procfs}, a hierarchical 
directory mounted at \texttt{`/proc/'}.
This directory is an interface
to kernel data and system information
that would otherwise be available
via system calls (if at all);
thus, many of the sources of nondeterminsm
already described can be propagated
through it.

For instance, \texttt{`/proc/uptime'} contains time statistics about how
long the system has been running;
\texttt{`/proc/meminfo'} contains statistics about kernel memory management;
\texttt{`/proc/net/'} contains statistics and information for system
network interfaces;
\texttt{`/proc/diskstats/'} contains statistics about any attached disks.
These files will differ across multiple executions
of a program because of nondeterminism in the underlying operating system. 

Apart from accessing system-wide information, a process can access 
information about its open file descriptors through
\texttt{`/proc/[PID]/fdinfo'} (e.g. cursor offsets and status).
Similarly, \texttt{`/proc/[PID]/status'} contains
process-specific and highly unpredictable statistics,
e.g. number of involuntary context switches,
memory usage, and parent process ID.
Performing a \texttt{stat} on files in \texttt{`/proc/[PID]/'}
can reveal the process creation time.

\subsection{Architecture Specific Instructions}
Architecture specific instructions such as \texttt{rdtsc}
and \texttt{cpuid} can return different
results across program executions. As mentioned before
(Section \ref{ch3:rand}), the \texttt{rdtsc} instruction provides the number of ticks since
the computer was last reset, which
will differ across executions. The \texttt{cpuid} instruction
can return conflicting hardware information too.

\section{Simulating Deterministic Execution} \label{ch3:simulation}



\section{Limitations of Deterministic Execution} \label{ch3:issues}
This section describes some of the drawbacks
of deterministic execution. \newline

\noindent \textbf{Security} \newline
To achieve deterministic execution,
we have to disable ASLR. We also have to fix 
canary (\texttt{gs:0x14}) or pointer guard (\texttt{gs:0x18})
 values across many different VMs.
Disabling ASLR increases the vulnerability of
applications to external attacks. 
Though canary and pointer guard
values are still dynamically chosen in
our brand of deterministic execution, 
they must agree across all VMs. Thus,
an adversary  who can compromise one VM by
guessing its canary could easily attack the other VMs.
The fact that we can choose different
canary or guard values between different successive bootstorms
is some consolation and provides some security. \newline

\noindent \textbf{Randomization} \newline
Randomization can be essential for security (e.g. random values
may be used to generate keys or certificates),
performance, and sometimes simply correctness (e.g. 
clients may choose random IDs for themselves).
Making PRNG seeds agree across all VM instances can entail
a compromise on all of these fronts.
Thankfully, we have not yet discovered any such
issues in the Linux services. Technically,
our approach simulates the extremely unlikely 
-- yet possible -- scenario that all concurrently executing
instances somehow generated the same seeds
from external sources. \newline

\noindent \textbf{Time and Correctness} \newline
Any programs that rely on precise
measurements of time (e.g. through \texttt{rdtsc})
will lose correctness. 
Some Linux services such as \texttt{ntp} 
do need to measure time accurately in order to synchronize the system clock
Our semantics can cause such services to 
behave incorrectly at start up,
because we may give incorrect values
of time to \texttt{ntp}.
Thankfully, this is not a huge correctness problem because network 
clock synchronization programs are self-healing and
because we ultimately do provide monotonically
 increasing time values. After the booting process is over,
and all VMs branch in execution, \texttt{ntp} will synchronize
the current time correctly. \newline

\noindent \textbf{I/O aggregation in Network} \newline
As indicated earlier, when file contents
or bytes read over sockets differ, 
these could be because of synthetic differences (e.g.
timestamps in headers), or because of 
semantic differences (e.g. different requests or data).
We can study application code and use dynamic instrumentation
to reconstruct file contents or network packets and overcome synthetic nondeterminism
from such sources. In our experiments, we used this approach
for \texttt{Netlink} packets and application files, but generalizing it
to all network sockets and protocols, while possible, would clearly
complicate the design of the dynamic instrumentation
layer. For semantic differences in I/O, 
execution would have to branch out.

One possible approach for fixing nondeterminism from external
socket \texttt{reads} would be to forcibly
conform these reads to be identical by replaying
them. This approach would work for many 
Linux services and would simulate the possibility
that these services received responses from
an external source at the exact same time containing
precisely the same data. However, these semantics can
also be problematic in terms of correctness for other services
e.g. if a network 
response essentially says {\em ``you have the lock''} or 
{\em ``your print job was successfully queued''} or
{\em ``your client id is 134''},
this approach falls apart.

\section {Summary}
This chapter described sources of non-determinism in Linux
services and ways to overcome them in detail.
It also discussed some of the limitations of deterministic
execution.
Chapter \ref{ch:case} describes the effectiveness
of the techniques described in this section
by presenting a case study of three Linux services.
